% This version of CVPR template is provided by Ming-Ming Cheng.
% Please leave an issue if you found a bug:
% https://github.com/MCG-NKU/CVPR_Template.

%\documentclass[review]{cvpr}
\documentclass[final]{cvpr}

\usepackage{times}
\usepackage{epsfig}
\usepackage{graphicx}
\usepackage{amsmath}
\usepackage{amssymb}

\usepackage{url}            % simple URL typesetting
\usepackage{booktabs}       % professional-quality tables
\usepackage{amsfonts}       % blackboard math symbols
\usepackage{nicefrac}       % compact symbols for 1/2, etc.
\usepackage{microtype}      % microtypography

\usepackage{graphicx}
\usepackage{amsmath,amssymb} % define this before the line numbering.
\usepackage{bbold}
\usepackage{dsfont}
\usepackage{enumerate}
\usepackage[toc,page]{appendix}
\usepackage[font=small,labelfont=bf]{caption}

\usepackage{enumitem}

\usepackage{xcolor}
\usepackage{caption}
% \usepackage{subcaption}
\usepackage{bm}
\usepackage{isomath}
\usepackage{arydshln}
\usepackage{stmaryrd}

% \usepackage{fixltx2e}
\usepackage{dblfloatfix}
\usepackage{pbox}
\usepackage{capt-of}

\usepackage[normalem]{ulem}
\usepackage{multirow}

\usepackage{colortbl}

\usepackage{mathtools}
\usepackage{array}

\makeatletter
\@namedef{ver@everyshi.sty}{}
\makeatother
\usepackage{pgf}

\newcommand{\xpt}{\edef\f@size{\@xpt}\rm}

\def\ie{\emph{i.e.}}
\def\etc{\emph{etc}}

\usepackage{tikz}

\newcommand{\comment}[1]{}

\usepackage{indentfirst}
\input{definitions.tex}

% Include other packages here, before hyperref.
\usepackage[numbib]{tocbibind}
% If you comment hyperref and then uncomment it, you should delete
% egpaper.aux before re-running latex.  (Or just hit 'q' on the first latex
% run, let it finish, and you should be clear).
\usepackage[pagebackref=true,breaklinks=true,colorlinks,bookmarks=false]{hyperref}


\def\cvprPaperID{8501} % *** Enter the CVPR Paper ID here
\def\confYear{CVPR 2021}
%\setcounter{page}{4321} % For final version only

\pagestyle{empty}
%\thispagestyle{empty}

\begin{document}

%%%%%%%%% TITLE
\title{Video Matting with Convolutional LSTM}
\author{\underline{Jiahao Zhang} \\ U6921098 \and Peng Zhang \\ U6921163 \and Hang Zhang \\ U6921112}

\maketitle
\begin{abstract}
    TODO
\end{abstract}


\section{Introduction}

background
motivation
related works
contribution summary

\section{Problem Statement}

definition, and formulation

\section{Methods}

\subsection{Model Structure}

\begin{figure*}[t]
    \begin{center}
        \includegraphics[width=1\textwidth]{img/modelConvLSTM.pdf}
    \end{center}
    \caption{The architecture of our network.}
    \label{modelConvLSTM}
\end{figure*}

\subsubsection{Encoder}
\subsubsection{ASPP}
\subsubsection{Decoder}

\subsubsection{ConvLSTM}

\begin{equation}
    \begin{aligned}
        \mathbf{i}_{\mathbf{t}} & = \operatorname{Sigmoid}\left(\operatorname{Conv}\left(\mathbf{x}_{\mathbf{t}} ; \mathbf{w}_{\mathbf{x i}}\right)+\operatorname{Conv}\left(\mathbf{h}_{\mathbf{t}-\mathbf{1}} ; \mathbf{w}_{\mathbf{h i}}\right)+\mathbf{b}_{\mathbf{i}}\right)    \\
        \mathbf{f}_{\mathbf{t}} & = \operatorname{Sigmoid}\left(\operatorname{Conv}\left(\mathbf{x}_{\mathbf{t}} ; \mathbf{w}_{\mathbf{x f}}\right)+\operatorname{Conv}\left(\mathbf{h}_{\mathbf{t}-\mathbf{1}} ; \mathbf{w}_{\mathbf{h f}}\right)+\mathbf{b}_{\mathbf{f}}\right)    \\
        \mathbf{o}_{\mathbf{t}} & = \operatorname{Sigmoid}\left(\operatorname{Conv}\left(\mathbf{x}_{\mathbf{t}} ; \mathbf{w}_{\mathbf{x o}}\right)+\operatorname{Conv}\left(\mathbf{h}_{\mathbf{t}-\mathbf{1}} ; \mathbf{w}_{\mathbf{h o}}\right)+\mathbf{b}_{\mathbf{o}}\right)    \\
        \mathbf{g}_{\mathbf{t}} & = \operatorname{Tanh} \quad\left(\operatorname{Conv}\left(\mathbf{x}_{\mathbf{t}} ; \mathbf{w}_{\mathbf{x g}}\right)+\operatorname{Conv}\left(\mathbf{h}_{\mathbf{t}-\mathbf{1}} ; \mathbf{w}_{\mathbf{h g}}\right)+\mathbf{b}_{\mathbf{g}}\right) \\
        \mathbf{c}_{\mathbf{t}} & = \mathbf{f}_{\mathbf{t}} \odot \mathbf{c}_{\mathbf{t}-\mathbf{1}}+\mathbf{i}_{\mathbf{t}} \odot \mathbf{g}_{\mathbf{t}}                                                                                                                           \\
        \mathbf{h}_{\mathbf{t}} & = \mathbf{o}_{\mathbf{t}} \odot \operatorname{Tanh}\left(\mathbf{c}_{\mathbf{t}}\right)
    \end{aligned}
\end{equation}\label{convLSTM}


\subsection{Loss Function}

\section{Experiments}

\subsection{Experiment Setup}

\subsubsection{Datasets}

\subsubsection{Metrics}

\subsubsection{Implementation}

\subsection{Experiment Results}

\subsubsection{Comparing}

\subsubsection{Ablation Study}

\section{Conclusion}

Conclusion \cite{chenRethinkingAtrousConvolution2017}

\section{References}

\bibliographystyle{ieee_fullname}
\bibliography{reference}

\section{Review}

\subsection{Self Reflection}

\subsection{Confidential Peer Review}

In doing this project, to the best of my judgement,
I confirm that Jiahao Zhang mainly contributed to TODO,
and his/her overall contribution is about 34\%,
Peng Zhang mainly worked on TODO,
and his/her contribution is about 33\%,
and Hang Zhang was responsible for TODO,
and his/her contribution counts about 33\% of the total project workload.

\end{document}
